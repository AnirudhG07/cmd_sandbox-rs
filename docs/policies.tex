\documentclass{article}
\usepackage{amsmath}
\usepackage{listings}
\usepackage{xcolor}
\usepackage{hyperref}
\usepackage{amssymb}
\usepackage{newunicodechar}
\usepackage{tabularx}
\newunicodechar{✅}{\checkmark}
\hypersetup{
    colorlinks=true,
    linkcolor=blue,
    filecolor=magenta,      
    urlcolor=cyan,
    pdftitle={User Command Sandbox Policy Requirements Document},
    bookmarks=true
}

\title{\bf User Command Sandbox: Policy Requirements Document}
\author{Teaching Assistant \\ E0256 - Computer Systems Security \\ Indian Institute of Science}
\date{Autumn 2025}

\begin{document}

\maketitle

\section{Introduction}
This document specifies the mandatory policy requirements for the User Command Sandbox project. All student implementations must enforce these policies at the kernel level for the \texttt{curl} command. The policies are designed to demonstrate the advantages of kernel-level enforcement over container-based approaches, providing fine-grained control over system resources and security boundaries.

\section{Policy Categories and Requirements}

\subsection{Network Access Policies}
\begin{table}[h]
\centering
\caption{Network Access Policy Requirements}
\label{tab:network_policies}
\begin{tabularx}{\textwidth}{|l|X|l|}
\hline
\textbf{Policy ID} & \textbf{Policy Description} & \textbf{Enforcement Level} \\
\hline
NET-001 & Allow HTTP/HTTPS connections only to domains specified in whitelist & BLOCK \\
NET-002 & Block all FTP, SFTP, and other non-HTTP protocols & BLOCK \\
NET-003 & Restrict maximum connection duration to 30 seconds & TIMEOUT \\
NET-004 & Limit concurrent connections to 3 simultaneous connections & THROTTLE \\
NET-005 & Block connections to private IP ranges (10.0.0.0/8, 172.16.0.0/12, 192.168.0.0/16) & BLOCK \\
NET-006 & Allow only ports 80 (HTTP) and 443 (HTTPS) & BLOCK \\
\hline
\end{tabularx}
\end{table}

\subsection{File System Policies}
\begin{table}[h]
\centering
\caption{File System Access Policy Requirements}
\label{tab:filesystem_policies}
\begin{tabularx}{\textwidth}{|l|X|l|}
\hline
\textbf{Policy ID} & \textbf{Policy Description} & \textbf{Enforcement Level} \\
\hline
FS-001 & Allow file writes only to \texttt{/tmp/curl\_downloads/} directory & RESTRICT \\
FS-002 & Block all file read operations outside user's home directory & BLOCK \\
FS-003 & Maximum file download size: 10MB per file & QUOTA \\
FS-004 & Prevent execution of downloaded files & BLOCK \\
FS-005 & Restrict total storage usage to 50MB & QUOTA \\
FS-006 & Block access to system directories (\texttt{/etc/, /bin/, /sbin/, /usr/}) & BLOCK \\
\hline
\end{tabularx}
\end{table}

\subsection{Memory and Process Policies}
\begin{table}[h]
\centering
\caption{Memory and Process Policy Requirements}
\label{tab:memory_policies}
\begin{tabularx}{\textwidth}{|l|X|l|}
\hline
\textbf{Policy ID} & \textbf{Policy Description} & \textbf{Enforcement Level} \\
\hline
MEM-001 & Maximum memory usage: 100MB & LIMIT \\
MEM-002 & Prevent fork() and exec() system calls during execution & BLOCK \\
MEM-003 & Maximum process execution time: 2 minutes & TIMEOUT \\
MEM-004 & Restrict CPU usage to 50\% of single core & THROTTLE \\
MEM-005 & Block memory mapping of executable pages & BLOCK \\
MEM-006 & Limit stack size to 8MB & LIMIT \\
\hline
\end{tabularx}
\end{table}

\subsection{Security and Isolation Policies}
\begin{table}[h]
\centering
\caption{Security and Isolation Policy Requirements}
\label{tab:security_policies}
\begin{tabularx}{\textwidth}{|l|X|l|}
\hline
\textbf{Policy ID} & \textbf{Policy Description} & \textbf{Enforcement Level} \\
\hline
SEC-001 & Run curl as non-privileged user (nobody) & ISOLATE \\
SEC-002 & Block access to environment variables containing "PASSWORD", "KEY", "SECRET" & FILTER \\
SEC-003 & Prevent network interface configuration changes & BLOCK \\
SEC-004 & Restrict signal handling (allow only TERM, INT) & FILTER \\
SEC-005 & Block access to kernel memory and modules & BLOCK \\
SEC-006 & Isolate network namespace from host & ISOLATE \\
\hline
\end{tabularx}
\end{table}

\section{Policy Configuration Format}
Students must implement policy configuration using JSON  (or any other suitable file format) as shown below:

\begin{lstlisting}[language=json, backgroundcolor=\color{gray!10}, basicstyle=\footnotesize\ttfamily]
{
  "policy_version": "1.0",
  "command": "curl",
  "network_policies": {
    "allowed_domains": ["example.com", "iisc.ac.in", "trusted.org"],
    "allowed_ports": [80, 443],
    "max_connections": 3,
    "connection_timeout": 30,
    "block_private_ips": true
  },
  "filesystem_policies": {
    "allowed_write_dirs": ["/tmp/curl_downloads/"],
    "max_file_size": 10485760,
    "max_total_storage": 52428800,
    "blocked_paths": ["/etc/", "/bin/", "/sbin/", "/usr/"]
  },
  "memory_policies": {
    "max_memory": 104857600,
    "max_stack_size": 8388608,
    "max_cpu_time": 120,
    "cpu_limit_percent": 50
  },
  "security_policies": {
    "run_as_user": "nobody",
    "blocked_environment": ["PASSWORD", "KEY", "SECRET"],
    "allowed_signals": ["TERM", "INT"],
    "isolate_network": true
  }
}
\end{lstlisting}

\section{Enforcement Mechanisms}

\subsection{System Call Interception}
Students must implement system call interception for the following critical operations:

\begin{itemize}
    \item \textbf{Socket operations:} socket(), connect(), bind(), accept()
    \item \textbf{File operations:} open(), openat(), read(), write(), mkdir()
    \item \textbf{Process operations:} fork(), execve(), clone()
    \item \textbf{Memory operations:} mmap(), brk(), mprotect()
    \item \textbf{Signal operations:} signal(), sigaction()
\end{itemize}

\subsection{Policy Violation Handling}
\begin{table}[h]
\centering
\caption{Policy Violation Response Requirements}
\label{tab:violation_handling}
\begin{tabularx}{\textwidth}{|l|X|X|}
\hline
\textbf{Violation Type} & \textbf{Required Action} & \textbf{Log Message} \\
\hline
Network Policy Violation & Block connection + Terminate process & "NETWORK\_VIOLATION: Attempted connection to blocked domain" \\
File System Violation & Block operation + Continue execution & "FS\_VIOLATION: Attempted write to restricted path" \\
Memory Limit Exceeded & Terminate process + Cleanup & "MEMORY\_VIOLATION: Exceeded allocated memory limit" \\
Timeout Violation & Terminate process & "TIMEOUT\_VIOLATION: Process exceeded maximum execution time" \\
Security Violation & Immediate termination & "SECURITY\_VIOLATION: Attempted privileged operation" \\
\hline
\end{tabularx}
\end{table}

\section{Testing and Validation Requirements}

\subsection{Mandatory Test Cases}
Students must demonstrate the following test scenarios:

\begin{enumerate}
    \item \textbf{Test NET-001:} Attempt to connect to non-whitelisted domain → Should be blocked
    \item \textbf{Test NET-005:} Attempt to connect to 192.168.1.1 → Should be blocked
    \item \textbf{Test FS-001:} Attempt to write to \texttt{/home/user/file} → Should be blocked
    \item \textbf{Test FS-003:} Download file larger than 10MB → Should be blocked
    \item \textbf{Test MEM-001:} Allocate 150MB memory → Process should be terminated
    \item \textbf{Test MEM-003:} Run process for 3 minutes → Should timeout and terminate
    \item \textbf{Test SEC-002:} Access environment variable with "PASSWORD" → Should be filtered
\end{enumerate}

\section{Evaluation Criteria}

\subsection{Policy Implementation (40\%)}
\begin{itemize}
    \item Complete implementation of all mandatory policies (20\%)
    \item Correct handling of policy violations (10\%)
    \proper Policy configuration parsing and application (10\%)
\end{itemize}

\subsection{Security Effectiveness (30\%)}
\begin{itemize}
    \item Successful prevention of all policy violations (15\%)
    \item Proper isolation from host system (10\%)
    \item Secure cleanup after termination (5\%)
\end{itemize}

\subsection{Code Quality and Documentation (30\%)}
\begin{itemize}
    \item Clean, well-documented kernel module/eBPF code (15\%)
    \item Comprehensive test suite coverage (10\%)
    \item Clear architecture documentation (5\%)
\end{itemize}

\end{document}